\documentclass{beamer}
\usepackage[utf8]{inputenc}
\usepackage{svg}
\usepackage{booktabs}
% \usepackage[maxcitenames=1]{biblatex}

\renewcommand{\footnotesize}{\tiny}
\setbeamertemplate{navigation symbols}{}
\pdfstringdefDisableCommands{%
  \def\\{}%
}

\title{Private Credit Fund - FX Hedging Strategies}
\institute{Validus Risk Management}
\author[Omar Nassef]{Omar Nassef}
\date{\tiny\today}

\usetheme{Warsaw}
\usecolortheme{beaver}

\setbeamercolor{block title}{use=structure,fg=black,bg=white!75!black}
\setbeamercolor{block body}{use=structure,fg=black,bg=white!20!white}

\setbeamertemplate{headline}{}

\begin{document}

\frame{\titlepage}

\begin{frame}
\tableofcontents
\frametitle{Table of Contents}
\end{frame}


\section{Executive Summary}
\begin{frame}{Executive Summary}

    \begin{block}{Problem}
        Fund faces currency mismatch. Contribution in USD but returns are in EUR.\\
        Performance becomes a function of underlying asset and currency movements.
    \end{block}
    
    \begin{block}{Solution}
        Static Forward Hedging for risk averse appetite.\\
        Dynamic Hedging Strategy for risk neutral with potential upside capture.
    \end{block}

    \begin{block}{Limitations}
        Interest rate simplification.\\
        Transaction costs omitted.
    \end{block}

\end{frame}

\section{Market Analysis}
\begin{frame}[allowframebreaks]{Market Analysis}
    
    \begin{figure}[H]
        \centering
        \includegraphics[width=0.9\linewidth]{../Figures/spot-rate.png}
        \caption{Currency Spot Rate}
        \tiny\textit{Key Dates: 2020 - COVID, 2022 - War Crisis}
    \end{figure}
    

    \framebreak

    \footnotesize
    \begin{table}
        \centering
        \begin{tabular}{c|c|c|c|c|c}
            \textbf{Regime} & \textbf{Freq} & \textbf{Spot Return} & \textbf{Volatility} & \textbf{Risk Reversal} & \textbf{Put Skew}\\
            \toprule
            High Vol (Panic) & 25\% & +2.2\% & 10.3\% & -134\% & 86\%\\
            Normal Vol & 50\% & +2.3\% & 7.0\% & -59\% & 77\%\\
            Low Vol (Calm) & 25\% & -3.7\% & 5.2\% & -31\% & 77\%\\
        \end{tabular}
        \\
        \footnotesize\textit{Risk Reversal: Difference between put and call volatility. -100\% = puts twice expensive as calls (fear).\\
        Uses 1 Year time-lag data.}
    \end{table}

    \normalsize

    \begin{block}{Panic Mode}
        Volatility Spikes to 2 $\times$ calm periods.\\
        Markets paying significant premium for EUR protection (-134\% RR).
    \end{block}

    \begin{block}{Calm}
        Low volatility but steady erosion of returns.
    \end{block} 

    Downside protection is disproportionately more expensive.
    
\end{frame}

\section{Methodology}
\begin{frame}{Methodology}
    
    \begin{block}{Model}
        Heston Stochastic Volatility (calibrated to historical data provided).
    \end{block}

    \begin{block}{Simulation}
        Monte-Carlo with 10,000 paths
    \end{block}

    \begin{block}{Assumptions}
        \begin{itemize}
            \item Static interest rates can be used to infer forward rates.
            \item Euler scheme for option price calculation is sufficient.
            \item Computational costs can be negated.
        \end{itemize}
    \end{block}

\end{frame}

\section{Core Findings}
\begin{frame}[allowframebreaks]{Core Findings}

    \begin{figure}[H]
    \centering
    \includesvg[width=0.8\linewidth]{../Figures/StrategyIRRDist.svg}
    \caption{IRR distribution for hedging strategies}
    \label{fig:irrdist}
\end{figure}

\framebreak

\begin{figure}[H]
    \centering
    \includesvg[width=0.8\linewidth]{../Figures/StrategyIRRScatter.svg}
    \caption{Mean Risk-Return Trade-off for hedging strategies}
    \label{fig:risk-return}
\end{figure}

\framebreak

\begin{figure}[H]
    \centering
    \includesvg[width=0.8\linewidth]{../Figures/StrategyMultipleDist.svg}
    \caption{Multiples on Invested Capital for hedging strategies}
    \label{fig:multiples}
\end{figure}

\framebreak

\begin{figure}[H]
    \centering
    \includesvg[width=0.8\linewidth]{../Figures/StrategyCVaR.svg}
    \caption{CVaR for hedging strategies}
    \label{fig:cvar}
\end{figure}
    
\end{frame}

\section{Tail Risk Analysis}
\begin{frame}[allowframebreaks]{Tail Risk Analysis}

    \begin{figure}[H]
    \centering
    \includesvg[width=0.8\linewidth]{../Figures/StrategyIRRDistExtreme.svg}
    \caption{IRR distribution for hedging strategies for extreme scenarios.}
    \label{fig:irrextreme}
\end{figure}

\framebreak

\begin{figure}[H]
    \centering
    \includesvg[width=0.8\linewidth]{../Figures/StrategyMultipleDistExtreme.svg}
    \caption{Multiples on Invested Capital for hedging strategies for extreme scenarios.}
    \label{fig:multiplesextreme}
\end{figure}
    
\end{frame}

\section{Cost-Benefit Analysis}
\begin{frame}{Cost-Benefit Analysis}

\tiny
\begin{table}[h]
    \centering
    \begin{tabular}{c|c|c|c|c}
         \textbf{Strategy} & \textbf{IRR Preservation} & \textbf{Risk Reduction} & \textbf{VaR Reduction}  & \textbf{Weighted Value}  \\
         \toprule
            No Hedge & 0  & 0 &0 & 0  \\
            Static Forward Hedge & -0.10 & 1.00 & 1.29 & 0.72 \\
            Partial Hedge 0.5 & -0.08& 0.54& 0.88& 0.45 \\
            Partial Hedge 0.8 & -0.10 & 0.79 & 1.14 & 0.61 \\
            Dynamic Delta Hedge & -0.14 & 0.71 & 1.29 & 0.61 \\
    \end{tabular}
    \caption{Weighted Cost-Benefit Analysis for the hedging strategies}
    \label{tab:cba}
\end{table} 

\normalsize
Weighted Value = equal weighting for each of the metrics.

\end{frame}

\section{Recommendations}
\begin{frame}{Recommendations}
    
    Static Forward Hedging:
    \begin{itemize}
        \item[$\checkmark$]  Premium cost minimised.
        \item[$\checkmark$]  Eliminates FX risk.
        \item[$\times$] No upside participation.
        \item[$\times$] Inflexible as cash flow changes.
        
    \end{itemize}
        
    Dynamic Hedging:
    \begin{itemize}
        \item[$\checkmark$] Upside movement participation. 
        \item[$\times$] Transaction costs from rebalancing.
        \item[$\times$] Lags during large quick delta moves. %Gamma
        \item[$\times$] Sudden volatility spikes may negatively impact option prices. %Vega
        % Vega measures an option's price sensitivity to changes in implied volatility.
        % Gamma measures the rate at which an option's Delta changes as the underlying asset's price moves.
    \end{itemize}
    
\end{frame}

\section{Limitation and Drawbacks}
\begin{frame}{Limitations and Drawbacks}

    \begin{itemize}
        \item Back testing not covered. Max drawdown, Sharpe ratio and tracking error missing as metrics. 
        \item Transaction Costs not taken into account. 
        \item Interest rate not modeled as a stochastic function.
        \item Partial Hedging numbers assigned arbitrarily.
        \item Unaware of client risk appetite, accounting needs or cost constraints.
    \end{itemize}
    
\end{frame}
    
\section{Next Steps}
\begin{frame}{Next Steps}

    \begin{itemize}
        \item Implement back-testing framework.
        \item Model interest rate as a stochastic function (e.g. Heston + SABR)
        \item Different weighting schemes depending on client needs.
        \item Modeling should be aware of regimes.
        \item Liquidity conscious decision-making and metrics (e.g. Liquidity VaR).
    \end{itemize}
    
\end{frame}

\end{document}
